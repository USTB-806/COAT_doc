% Give your report a title
\newcommand\reporttitle{计算机组成原理实验指导书附录}

% Insert course code, name, quartile number and year (or any other subtitle)
\newcommand\reportsubtitle{
    2024, Fall
}

% Add your group number (for DBL) or any other text.
%\newcommand\groupnumber{
%    \textbf{Group 0}
%}

% Insert authors and student numbers here
\newcommand\reportauthors{
    计算机组成原理课程组 \\
}

% Add the name of your tutor (for DBL) or any other text.
\newcommand\grouptutor{
    Tutor: Name Surname
}

% Date and location (default: current date and Eindhoven)
\newcommand\placeanddate{
    USTB, \today
}

% Define Ustb-blue (color of the TU/e logo). Can be changed to drastically change the look of the template
%\definecolor{Ustb-blue}{RGB}{199, 25, 24}
\definecolor{Ustb-blue}{RGB}{44, 84, 149}

% All of the following code can be removed to be left with (close to) default LaTeX behaviour.

% Sets up hyperlinks in the document to be colored
\hypersetup{
    colorlinks=true,
    linkcolor=Ustb-blue,
    urlcolor=Ustb-blue,
    citecolor = Ustb-blue
}
\urlstyle{same} % Defines settings for link and reference formatting


% Change bullet style for level 1, 2 and 3 respectively for itemize
\renewcommand{\labelitemi}{\scriptsize\textcolor{Ustb-blue}{$\blacksquare$}}% level 1
\renewcommand{\labelitemii}{\scriptsize\textcolor{Ustb-blue}{$\square$}}% level 2
\renewcommand{\labelitemiii}{\textcolor{Ustb-blue}{$\circ$}}% level 3

% \renewcommand{\labelitemi}{\small\textcolor{Ustb-blue}{\ding{70}}} % level 1
% \renewcommand{\labelitemii}{\small\textcolor{Ustb-blue}{\ding{71}}}% level 2
% \renewcommand{\labelitemiii}{\tiny\textcolor{Ustb-blue}{\ding{71}}}% level 3

% Change bullet style for level 1, 2 and 3 respectively for enumerate
\renewcommand{\labelenumi}{\textbf{\textcolor{Ustb-blue}{\arabic*.}}}% level 1
\renewcommand{\labelenumii}{\textbf{\textcolor{Ustb-blue}{[\alph*]}}}% level 2
\renewcommand{\labelenumiii}{\textbf{\textcolor{Ustb-blue}{\roman*.}}}% level 3

% Have reference labels be linked to section (section 3 will have fig. 3.1 etc.)
\counterwithin{equation}{section} % For equations
\counterwithin{figure}{section} % For figures
\counterwithin{table}{section} % For tables

% Creates a beautiful header/footer
\pagestyle{fancy}
\lhead{\includegraphics[height = 8pt]{Figures/0. General/ustb_logo_blue_small.pdf}}
\rhead{\reporttitle}
\renewcommand{\footrulewidth}{0.4pt}
\cfoot{Page \thepage}

% Formats section, subsection and subsubsection titles respectively
\titleformat{\section}{\sffamily\color{Ustb-blue}\Large\bfseries}{\thesection\enskip\color{gray}\textbar\enskip}{0cm}{} % Formats section titles

\titleformat{\subsection}{\sffamily\color{Ustb-blue}\large\bfseries}{\thesubsection\enskip\color{gray}\textbar\enskip}{0cm}{} % Formats subsection titles

\titleformat{\subsubsection}{\sffamily\color{Ustb-blue}\bfseries}{\thesubsubsection\enskip\color{gray}\textbar\enskip}{0cm}{} % Formats subsubsection titles

% Formats captions
\DeclareCaptionFont{Ustb-blue}{\color{Ustb-blue}}
\captionsetup{labelfont={Ustb-blue,bf}}

% Changes font to mlmodern
\usepackage{mlmodern}
\usepackage{color}
\usepackage{graphicx}

% Removes indent when starting a new paragraph
\setlength\parindent{0pt}

% Limits the ToC to sections and subsections (no subsubsec.)
\setcounter{tocdepth}{2}

\setlength{\parindent}{2em}

\setmonofont{Fira Code}

\lstset{
    language=Verilog,
    backgroundcolor = \color{Ustb-blue!10},    % 背景色:淡黄
    basicstyle = \small\ttfamily,           % 基本样式 + 小号字体
    rulesepcolor= \color{gray},             % 代码块边框颜色
    breaklines = true,                  % 代码过长则换行
    numbers = left,                     % 行号在左侧显示
    numberstyle = \small,               % 行号字体
    keywordstyle = \color{Ustb-blue}\textbf,            % 关键字颜色
    commentstyle =\color{purple},        % 注释颜色
    stringstyle = \color{Ustb-blue!50!purple},          % 字符串颜色
    frame = shadowbox,                  % 用(带影子效果)方框框住代码块
    showspaces = false,                 % 不显示空格
    columns = fixed,                    % 字间距固定
    morekeywords = {as},                % 自加新的关键字(必须前后都是空格)
    deletendkeywords = {compile}        % 删除内定关键字;删除错误标记的关键字用deletekeywords删!
}

\newcommand{\foreword}[4]{
    \begin{minipage}[l]{0.4\textwidth}
        \textit{\textcolor{Ustb-blue}{\Large #1}}\normalsize \textit{(#2)}\\
        {\small\textit{#3}}
    \end{minipage}
    \hfill
    \begin{minipage}[r]{0.55\textwidth}
        \begin{flushright}
            \includegraphics[width=.9\textwidth]{#4}
        \end{flushright}
    \end{minipage}
    \vspace{1em}
    \textcolor{gray}{\[\sim \cdot \sim\]}
    \vspace{1em}
}

\newcommand{\highlight}[1]{
    \textbf{\textcolor{Ustb-blue}{#1}}
}

\setcounter{tocdepth}{4}
\setcounter{secnumdepth}{3}