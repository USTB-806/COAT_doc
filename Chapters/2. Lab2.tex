\section{访存指令的添加}

\foreword{
    跨越阿尔卑斯山圣伯纳隘道的拿破仑
}{
    雅克·路易·大卫
}{
    %    不犯错误,那是天使的梦想。尽量少犯错误,这是人的准则;错误就像地心具有吸引力,尘世的一切都免不了犯错误。
    就像一只回旋镖,你已经忘记曾经把他掷出,却在意想不到的时候飞回你手中。
}{
    Figures/C2.jpg
}

\subsection{\texttt{st.b}和\texttt{st.h}指令的添加}

实现的关键在于控制如何实现字节或者半字的写入。

显然是通过字节写使能来实现的。例如\texttt{st.b} 是写入一个字节的数据, 所以他的写使能需要根据\texttt{vaddr}的最后两位进行判断。


\begin{verbatim}
    00 -> 0001
    01 -> 0010
    10 -> 0100
    11 -> 1000
\end{verbatim}

\texttt{st.h}指令同理,最后只需要生成写数据即可。

\begin{verbatim}
    00 -> 0011
    10 -> 1100
\end{verbatim}

\subsection{\texttt{ld.b}、\texttt{ld.h}、\texttt{ld.bu}、\texttt{ld.hu} 指令的添加}

分析与\texttt{ld.w}指令之间的异同:
\begin{itemize}
    \item 他们之间的计算虚地址,操作数来源,虚实地址映射的方法规则是完全一样的
    \item 访存结果都写回rd项寄存器
    \item 差异仅仅是写回的数据的位宽是不同的
\end{itemize}

这四条指令的设计可以参考\texttt{ld.w}指令,并无太大的难度,唯一需要考虑的是如何从数据RAM中选择出所需要的内容,
可以通过一个多路选择器去实现,选择信号是根据指令的访存地址的\highlight{最低两位}以及访存操作的类型信息共同决定的。
可结合上面的\texttt{st.b}和\texttt{st.h}指令的添加进行理解实现,两者的实现方法大同小异。

\highlight{不要忘记将取回内容扩展至32位,记得判断是符号扩展还是零扩展}
