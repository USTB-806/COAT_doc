\section{写在前面} \label{section: introduction}

\foreword{
    第一步
}{
    文森特·梵·高
}{
    人们经常会忘记的是,纵使是一趟没有目的地的旅程,也还是从单纯踏出第一步开始的。
}{
    Figures/C0.jpg
}

\begin{flushright}
    The machine is cold, but COAT is warm.

    \highlight{Computer Organization and Architecture Tutorial}

    Welcome to the digital world!
\end{flushright}

\vspace*{2em}

计算机组成原理课程,作为计算机专业的基础课,是计算机专业的使用者理解计算机构成的原理和学会用好计算机的关键。在这门课程当中,你会了解到一个基本的计算机系统是如何构成,了解到软件程序与硬件交互,更进一步地了解计算机系统背后的原理。

\subsection{关于RTFM, STFW, RTFSC}

在学习的过程中,你会使用到我们提供的处理器开发环境,其中包括了对基本的处理器的仿真验证和综合上板的功能。在这一过程中,你可能会遇到一些错误,这些错误,
有些可能是由于你对实验内容的不够了解,有些可能是因为你对实验环境提供的工具甚至是实验环境本身的不了解,这些问题不仅是在实验当中,也可能存在于大家未来的学习和工作当中,基于此,我们在这里向大家介绍计算机专业前辈们总结出对付错误的经验。

\paragraph{\bfseries RTFM} 阅读手册。在本实验中提供的手册有很多,比如实验指导书,比如说关于指令集架构的文档。如果你遇到错误,请首先先翻一翻手册,
看看是否是因为手册中某个没有注意到细节而导致了错误。手册会对实验环境和实验内容进行简要的介绍,它无法覆盖实验中的方方面面,
但是一定会将实验中的重点难点展现给大家,因此查询手册往往能够帮助我们快速解决问题。我们会在后面的资源使用路线中简单介绍一下本课程中可能会用到的各手册的内容。

\paragraph{\bfseries STFW} 上网搜索。手册会尽量将实验中的细节描述清楚,帮助大家完成实验。然而遗憾的是,手册没有办法覆盖到实验中的每一个细节,每一种可能的错误,
(比如大魔王Vivado的各种报错)所以这种时候可以考虑上网搜索。从效率的角度考虑,我们希望大家尽量使用Google/Bing 国际版进行搜索,多在国外的一些优秀论坛上寻找解决方案。

\paragraph{\bfseries RTFSC} 阅读源代码。实验环境中有大量代码,他们是整个实验环境的支撑。想要完成好实验,一定需要充分的了解实验环境,手册中已经给出了一部分关于实验环境的描述,
但是并不足以支持你理解实验环境的全貌,所以适当的阅读实验环境中的源代码,诸如测试用例的源代码,仿真环境的源代码,通过阅读这些代码,相信你不仅能解决问题,也能加深自己对于整个实验环境的理解。

当自己已经充分尝试过以上方法时,你可以向助教,老师,或者是其他同学求助,但是希望你在提问的过程中,能够尽量清晰的描述你的问题,帮助你提问的人能够更快理解你当前所面对的情况,更好的解决问题。

关于如何提问,你可以阅读《提问的智慧》或《别像弱智一样提问》,上面的三板斧和正确提问的方法是计算机世界的答案之书,在本门课程的理论内容的学习之外,我更希望大家能够学会正确的提问,学会正确而高效的寻找答案的方法。