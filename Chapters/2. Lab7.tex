\section{附加实验}

\foreword{
    罗纳河上的星夜
}{
    文森特·梵·高
}{
    人类企图攀及星辰的高度,在彩色玻璃和石块上镂刻下自己的事迹。
}{
    Figures/C7.png
}


\subsubsection{实现简易电子秤}\label{sss: 5.4}
实现一个具有基础功能的电子秤,具体要求如下:

\begin{enumerate}
    \item 电子秤功能:能计算本次价格与累计价格,并将其显示在数码管上。\footnote{例如:本次为第3次输入累加,单价为3,质量为2,历史总价为20,则数码管可以显示:“AC03 0026”(总输入次数与总价)。按下某个按键后,数码管的显示内容切换为:“0302 0006”(本次的单价、质量与价格)。}
          \begin{enumerate}
              \item 单次计价:输入物品的重量、单价,显示物品的总价(=重量*单价)。
              \item 累计计价:
                    \begin{enumerate}
                        \item 可以在Ego1板上设置累计按键
                        \item 按下累计按键时,记住当前物品的总价(假设当前物品记为物品1),可采用数码管依次显示以下信息:
                              \subitem \verb|AC 次数 应付总价|
                        \item 继续输入物品2的重量、单价,显示物品2的总价。按下累计按键,将本次物品2的总价累加进之前的用户应付总价中,数码管显示:
                              \subitem \verb|AC 次数 总价|
                        \item 依次购买物品3、4……,每个物品后都通过累计按键将本次物品总价累计到应付总价中
                        \item 进入累计状态的方式可以自由设计。
                    \end{enumerate}
              \item 退出累计状态:按下清除累计按键,恢复普通状态。
          \end{enumerate}
    \item 通过UART接口实现电脑端和Ego1的数据传输,要求使用电脑的串口传输软件(例如压缩包中提供的XCOM)将重量、单价等数据传输给电子秤,电子秤进行相应的计算、累加、存储等处理,电子秤将需要的反馈结果传输回电脑端(回传信息可自由设计)。
    \item 该电子秤具有帧解析功能:需要设计数据传输的控制协议,规定重量、单价等不同含义的数据的传输格式,根据该协议电子秤能够解析通过UART协议接收的数据桢(识别出重量、单价等不同含义的数据),并对其存储、处理。
    \item 以上是必备功能要求。根据我们提供的帧格式,你可以使用最大4个字节来存储单价和质量,鼓励进一步实现定点小数的输入与计算功能,例如单价是12.34,质量是11.11,那么输出的本次价格为137.10(四舍五入至2位小数)。具体如何实现,答案在你们的创造力中。
    \item 该电子秤具有清零功能,按下清零键后,一切恢复如初。数码管应有对应信息输出,例如“CLR”。
    \item 该电子秤能在每次输入次数或总价发生改变时(包括清零)通过UART向电脑端发送数据,数据包括输入次数、单价、质量、本次价格、总价。电脑端解析该数据,并将其输出在电脑屏幕上。比如要求1中的例子来说,可以输出: “AC03 0026 0302 0006”
          \subitem 然而如果只是单纯地输出数据的话,未免有些过于朴素了。鼓励通过整齐的格式给出完整的信息。一位助教Kevin给出的输出例子是:
          \begin{lstlisting}
        Vanity Fair's Balance
        price: 1.0$
        amount: 2
        ---accumulate mode begin---
        price: 3$
        amount: 4
        price: 12.34$
        amount: 56
        ---accumulate mode end---
        total: xx.xx$
        Vanity Fair's Balance
    \end{lstlisting}
          具体的输出格式实现,大家可以自由发挥
    \item 设计具有开放性,鼓励设计实现自己的输入输出控制逻辑和显示效果。也可自行实现帧协议,或扩展本实验提供的帧协议。
\end{enumerate}
