\section{MIPS组评分细则}

\subsection{任务要求}

\subsubsection{写在开头}

\highlight{个人任务需要个人独立完成验收,创新扩展部分在答辩环节进行展示,需要在答辩中展示创新扩展的正确性。}

\begin{itemize}
    \item 关于答辩会提的问题:
    \begin{itemize}
        \item 主要会围绕着创新内容部分进行考察,考察原理和实现思路。
    \end{itemize}
    \item 关于评分与难度的非线性关系:
    \begin{itemize}
        \item 之所以存在这种非线性关系,主要是为了一般的同学能够拿到分数,对此感兴趣的同学可以深入。不喜欢这个方向就没必要卷计组课设了,抓紧时间学好其他专业课。
    \end{itemize}
    \item 关于分数分布
    \begin{itemize}
        \item 本课程的总分100分,难度最大的部分其实在创新扩展部分,而其他的部分都只要认真把握课程原理,即可实现,请同学们重点把握这部分知识,充分理解;
        \item 不建议大家死磕创新扩展的实现,\highlight{更禁止通过抄袭的方式获得这一部分分数},请大家实事求是。
    \end{itemize}
\end{itemize}

\subsubsection{个人任务 60分}

\paragraph{CG 测评任务 20分}

完成CG上评测题,需要提交的代码已经在CDE工程中给出。请阅读实验指导书《计算机组成原理课程设计实验指导书 下》,学习TinyMIPS的基本结构。并提交相应代码。

\paragraph{虚拟仿真实验平台 10分}

\begin{itemize}
    \item 完成虚拟仿真实验平台上的四个实验。包括模拟器运行和代码运行;
    \item 本部分实验重点在于解决流水线的前递与暂停问题,与实验指导书的第二章结尾和第三章结尾有关。请注意体会其原理,在进行乘除法器的扩展过程中,同样需要利用到前递和暂停的知识。
\end{itemize}

\paragraph{扩展指令任务 20分}

\begin{itemize}
    \item 要求选择TinyMIPS CPU中未扩展的22条指令中的若干条进行扩展,具体包含的指令在表~\ref{tab: x1},对于ADD,SUB等涉及异常的指令,暂时可以不用扩展异常处理的内容。
    \item 对于每一条指令,都有若干其所属的分组和一个基础分值。
    \item 每个指令的基础分值根据扩展该指令时的工作量进行设置,具体见表~\ref{tab: x1}。
    \item 如果选择了多条同分组的指令,自第二条起,其计算分值减少为基础分值的1/2。
    \item 要求最终同学们完成计算分值总计10分的指令扩展。多做不加分。
    \item 个人任务需要到现场进行验收,演示对指令的测试过程。验收现场会进行提问,提问内容不仅涉及下面的指令扩展的思路过程,也包括Trace机制,测试用例,MIPS指令集架构相关的知识。
\end{itemize}


\begin{table}
    \centering
    \caption{指令分值}\label{tab: x1}
    \begin{tabular}{cccc}
        \toprule
        {\bfseries 序号} &  {\bfseries 指令名称} &  {\bfseries 指令分值} &  {\bfseries 指令分组}\\ 
        \midrule
        1       &   ADD     &   2       &       R型运算指令/加法指令        \\
        2       &   SUB     &   2       &       R型运算指令     \\
        3       &   NOR     &   2       &       R型运算指令     \\
        4       &   ADDI    &  2        &       I型运算指令/加法指令        \\
        5       &   ANDI    &  2        &       I型运算指令     \\
        6       &   ORI     &   2       &       I型运算指令     \\
        7       &   XORI    &  2        &       I型运算指令     \\
        8       &   SLTI    &  2        &       I型运算指令     \\
        9       &   SLTIU   &     2     &       I型运算指令     \\
        10      &   SRL     &   2       &       Shamt移位指令       \\
        11      &   SRA     &   2       &       Shamt移位指令       \\
        12      &   J   &     2     &       直接跳转指令        \\
        13      &   JR  &    3      &       直接跳转指令        \\
        14      &   LH  &    3      &       内存载入指令        \\
        15      &   LHU     &   3       &       内存载入指令        \\
        16      &   SH  &    3      &       内存存储指令        \\
        17      &   BGEZ    &  3        &       条件分支指令/BGE跳转        \\
        18      &   BGTZ    &  3        &       条件分支指令        \\
        19      &   BLEZ    &  3        &       条件分支指令        \\
        20      &   BLTZ    &  3        &       条件分支指令/BLT跳转        \\
        21      &   BGEZAL  &    3      &       条件分支指令/BGE跳转        \\
        22      &   BLTZAL  &    3      &       条件分支指令/BLT跳转        \\
        \bottomrule
    \end{tabular}
\end{table}

例如一个同学选择了ADD指令,ORI指令,ANDI指令,JR指令,LH指令,其总分值为2,2,1,3,3 共计 11分,可以达到验收要求。

\paragraph{实验报告——个人部分 10分}

\subsubsection{团队任务 40分}

\paragraph{创新基础部分}

创新部分基础实现包括《计算机组成原理课程设计指导书补充内容》所介绍的全部内容:

\begin{itemize}
    \item 乘除法指令扩展:
    \begin{itemize}
        \item 通过乘除法测试点,包括\texttt{n44\_div},\texttt{n45\_divu},\texttt{n46\_mult},\texttt{n47\_multu},\texttt{n48\_mfhi},\texttt{n49\_mflo},\texttt{n50\_mthi},\texttt{n51\_mflo};
        \item 乘除法模块可以采用IP核实现。
    \end{itemize}
    \item 异常相关指令扩展:
    \begin{itemize}
        \item 第一阶段:
        \begin{itemize}
            \item 实现特权指令\texttt{ERET},\texttt{MTC0},\texttt{MFC0}
            \item 自陷指令\texttt{BREAK},\texttt{SYSCALL}
            \item 实现A02文件中规定的所有CP0寄存器
        \end{itemize}
        \item 第二阶段:
        \begin{itemize}
            \item 继续扩展异常,通过完整89个功能点的测试
        \end{itemize}
    \end{itemize}
\end{itemize}

\paragraph{创新扩展部分}

给出多个选项参考,并不需要大家全部完成,如有其他自行设计的创新扩展不在本范围内也可。

\begin{itemize}
    \item 缓存模块设计:
    
    Cache是现代计算机中不可缺少的技术。

    \begin{itemize}
        \item Cache基本实现为一路直接映射Cache,二者都实现时取DataCache成绩;
        \item Cache指令实现;
        \item 多路组相连Cache;
        \item Cache流水化。
    \end{itemize}
    \item 外设功能扩展
    \begin{itemize}
        \item 通过汇编语言代码对SoC上已经提供实现了的外设进行操控;
        \item 集成串口外设,并能通过汇编语言代码操纵外设。
    \end{itemize}
\end{itemize}

\paragraph{实验报告——团队部分 10分}



