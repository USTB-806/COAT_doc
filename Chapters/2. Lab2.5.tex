\section{算术逻辑转移类指令的添加}

\foreword{
    僧侣步行到阿索斯山修道院
}{
    赫尔曼·科洛迪
}{
    路途中的一切,有些与我擦肩而过从此天各一方,有些便永久驻进我的心魂,雕琢我,塑造我,锤炼我,融入我而成为我。
}{
    Figures/C9.png
}

添加指令的设计过程:

\begin{itemize}
    \item 认真阅读指令系统规范,明确待实现指令的功能定义。
    \item 根据指令的功能定义考虑数据通路的设计调整,能复用的尽量复用,该新增的就新增。
    \item 根据调整后的数据通路,梳理所有指令(包括原有的指令和新增的指令)对应的控制信号。
\end{itemize}

需要添加的指令为

\texttt{slti},
\texttt{sltui},
\texttt{andi},
\texttt{ori},
\texttt{xori},
\texttt{sll.w},
\texttt{srl.w},
\texttt{sra.w},
\texttt{pcaddu12i},
\texttt{blt},
\texttt{bge},
\texttt{bltu},
\texttt{bgeu}.

通过比较已经完成的20条指令,不难发现,这些指令在各个阶段均可以复用原本已经实现的指令的数据通路,
然后对流水线上的所需使用的控制信号进行调整,增加部分信号即可,不在此进行过多的描述。

\subsection{实验要求}

\begin{itemize}
    \item 添加算术逻辑运算类指令slti、sltui、andi、ori、xori、sll、srl、sra、pcaddu12i。运行对应的func,要求成功通过仿真和上板验证.
    \item 添加转移指令blt、bge、bltu、bgeu,运行对应的func,要求成功通过仿真和上板验证。
\end{itemize}