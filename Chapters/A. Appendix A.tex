\section{LA组评分细则}

\subsection{任务要求}

\subsubsection{写在开头}

\highlight{个人任务需要个人独立完成验收,创新扩展部分在答辩环节进行展示,需要在答辩中展示创新扩展的正确性。}

\begin{itemize}
    \item 关于答辩会提的问题:
    \begin{itemize}
        \item 主要会围绕着创新内容部分进行考察,考察原理和实现思路。
    \end{itemize}
    \item 关于评分与难度的非线性关系:
    \begin{itemize}
        \item 之所以存在这种非线性关系,主要是为了一般的同学能够拿到分数,对此感兴趣的同学可以深入。不喜欢这个方向就没必要卷计组课设了,抓紧时间学好其他专业课。
    \end{itemize}
    \item 关于分数分布
    \begin{itemize}
        \item 本课程的总分100分,难度最大的部分其实在创新扩展部分,而其他的部分都只要认真把握课程原理,即可实现,请同学们重点把握这部分知识,充分理解;
        \item 不建议大家死磕创新扩展的实现,\highlight{更禁止通过抄袭的方式获得这一部分分数},请大家实事求是。
    \end{itemize}
\end{itemize}

\subsubsection{个人任务 60分}

根据给出的实验指导书完成前五个章节:

\begin{itemize}
    \item 不考虑冲突的简单五级流水划分;
    \item 指令相关和流水线冲突;
    \item 算术逻辑转移类指令的添加;
    \item 访存指令的添加;
    \item 乘除法指令的添加。
\end{itemize}

\paragraph{实验报告——个人部分 10分}

\subsubsection{团队任务 40分}

\paragraph{创新基础部分}

异常相关指令扩展 (根据指导书的最后一章,实现异常和中断的支持):

\begin{itemize}
    \item 第一阶段:
    \begin{itemize}
        \item 增加 csrrd、csrwr、csrxchg 和 ertn 指令;
        \item 为 CPU 增加控制状态寄存器 CRMD、PRMD、ESTAT、ERA、EENTRY、SAVE0-3;
        \item 为 CPU 增加 syscall 指令,实现系统调用异常支持。
    \end{itemize}
    \item 第二阶段:
    \begin{itemize}
        \item 增加取指地址错(ADEF)、地址非对齐(ALE)、断点(BRK)和指令不存在(INE)异常的支持;
        \item 增加中断的支持,包括 2 个软件中断、8 个硬件中断和定时器中断;
        \item 增加 rdcntvl.w、rdcntvh.w 和 rdcntid 指令;
        \item 增加控制状态寄存器 ECFG、BADV、TID、TCFG、TVAL、TICLR。
    \end{itemize}
\end{itemize}

\paragraph{创新扩展部分}

给出多个选项参考,并不需要大家全部完成,如有其他自行设计的创新扩展不在本范围内也可。

\begin{itemize}
    \item AXI总线接口设计:
    \begin{itemize}
        \item 将CPU顶层接口修改为AXI总线接口。
    \end{itemize}
    \item 缓存模块设计:
    
    Cache是现代计算机中不可缺少的技术。

    \begin{itemize}
        \item Cache基本实现为一路直接映射Cache,二者都实现时取DataCache成绩;
        \item Cache指令实现;
        \item 多路组相连Cache;
        \item Cache流水化。
    \end{itemize}
    \item 外设功能扩展
    \begin{itemize}
        \item 通过汇编语言代码对SoC上已经提供实现了的外设进行操控;
        \item 集成串口外设,并能通过汇编语言代码操纵外设。
    \end{itemize}
\end{itemize}

\paragraph{实验报告——团队部分 10分}



